This paper studies the theoretical properties of a set of well-established panel models compared to the novel GFE model.  All estimators impose a different set of assumptions on the data. The GFE which is the main focus of this paper, is a flexible approach, well-adapted to account for clustered unobserved heterogeneity impacting regressors and the dependent variable. GFE's assumptions include small unobserved individual heterogeneity within groups and a known number of groups. 
Simulations compare the methods discussed in the theoratical part when a grouped pattern of unobserved heterogeneity exists. A tailored DGP for GFE looks at its finite sample properties compared to well-established methods. 

After establishing the validity of employing GFE model, a more detailed simulation study is conducted for GFE estimator with misspecified groups. I provide heuristics on model selection from OLS and IFE. 
Simulation results do not suffer from large finite sample inefficiencies when the number of groups are overspecified. This suggests the inference on common parameters is still possible with overspecified groups. Even though grouped-fixed effect estimates suffer from large biases.

The similar finite sample properties of GFE with overspecified number of groups and GFE estimator with true number of groups suggest a possibility for asymptotic equivalence of the two estimator. Showing the asymptotic equivalence of the GFE estimator with overspecified number of groups and GFE estimator with true number of groups pose  a challenge and an interesting question for future research econometric theory.
