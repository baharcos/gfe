\subsection{Evaluation of the Computation Method of GFE} \label{section:sensitivity}
\paragraph{Sensitivity.} The iterative algorithm described in subsection \ref{section:algo} depends on the starting values influencing whether the algorithm converges to a local minima or the global minima.Practical solution suggested by \textcite{bonhomme2015grouped} for low dimensional problems is to drawing starting values at random and selecting the solution yielding the lowest objective. This additional procedure increases the computational time of the algorithm significantly.It has not been implemented due to time constraints and to shorten the amount of time the simulations take. However, to ensure the quality of the results of my implementation, the sensitivity to starting values has been studied.

In the simulation tests for the sensitivity to starting values, the algorithm was insensitive to values of $\theta^{(0)}$ given a reasonable value is chosen\footnote{Extreme cases such as $\theta^{(0)}$ is outside the range of the outcome variable is not tested}. In contrast, it is sensitive to $\alpha^{(0)}$ values. 
Given the DGP in subsection \ref{section:DGP}, diverting the true $\alpha$ values up to a 0.1 change, results in 24 percent of different objective functions and thus different estimates. The mean of $\hat{\theta}$ estimates from 100 simulation runs is different only after the third decimal point such. The tests also show that using starting values does not necessarily lead to the global minima.  

The starting values for different groups should be also well separated in order to avoid singular matrix problems occurring when the solution algorithm has an empty group and the GFE estimates are not identified. This has been a prominent issue in earlier simulation runs.How much variation in error term the algorithm and the estimator allows worth further attention. \textcite{bonhomme2015grouped} suggest as a solution to assign one random individual into the empty group which decreases the objective function.

\paragraph{Computational performance in comparison to other methods.}
I used existing established packages for the remaining methods. The computation was fastest for OLS but followed up by GFE and slowest for IFE. Nevertheless, it should be noted that established packages runs various checks and often compute a regression table with statistics that are not always needed for the simulation results. 

The code of my implementation of the estimator, simulations, and the additional results which are omitted here can be found on: \href{https://github.com/baharcos/gfe.}{github.com/baharcos/gfe}.
