Panel data contains multiple observations for each individual unit over time. Combining the elements of cross-section and time-series data allows to exploit time dynamic relationships, and to capture unobserved heterogeneity. This section describes theoretical properties of a number of linear panel models and their respective estimators.  %and use the OLS estimation method.

Denote a general panel model as
\begin{align} \label{eqn:general}
y_{it} = x_{it}'\theta + \mu + c_i + \xi_t + \lambda_i' f_t + \alpha_{g_{i}t} + \upsilon_{it},
\end{align}
where 
\begin{align} \label{eqn:regressor}
x_{it} = a(c_i) +  b(\xi_t) +  h(\lambda_i' f_t) + d (\alpha_{g_{i}t}) + \eta_{it},
\end{align}
for $i = 1,...,N, t = 1,...,T, g = 1,..., G.$  

$a, b, d, h$ are functions of $c_i, \xi_t, \alpha_{g_{i}t}, \lambda_i' f_t$ respectively and they do not impose a functional form restriction. %do not impose a functional form restriction. ,
$\mu$\footnote{The intercept will not be included in the panel models with unobserved heterogeneity discussed here as various dummy variables included to account for the heterogeneity, or a factor structure is fitted. An intercept is not necessary when all dummy variables are included. However, the dummy variables then carry the intercept; therefore, most standard FE and IFE packages estimate it.} is the intercept common across all groups, individuals and periods, $c_i$ is a time-invariant unit fixed effect, $ \xi_t$ is a unit-invariant time trend, $f_t$ is a $1\times r$ vector of period specific common shocks and $\lambda_i$ is a $1\times r$ vector to account for individual-specific sensitivity of the common shocks in $\lambda_i' f_t$, $\alpha_{g_{i}t}$ is a dynamic grouped-fixed effect. All of these are unobserved to the researcher. The presence of these different sources of unobserved heterogeneity complicates inference on model parameters with data. Each of the unobservable parts change the relationship between unobserved and observed variables. Therefore, there is a need for different identification strategies via tailored estimation methods.
%there is too way to look at it, restrict the regressors dgp or restrict the structure of the error term. Both are equivalent. We will restrict the error term structure.
We use the most simple model where unobserved variables do not cause any omitted variable bias as the baseline model. We use this model to understand the pooled OLS estimator and its requirements. Subsequently, we include the unobserved variables in (\ref{eqn:general}) one by one allowing for various heterogeneity patterns and examine estimators proposed in the literature, respectively.
%In all models, we consider a balanced panel where each cross-section unit has the same number of observations over time. All estimators minimize an LS criterion.
I use \textcite{hansen2022econometrics} and \textcite{wooldridge2010econometric} as the combined source for the theoretical properties of pooled ordinary least squares (pooled OLS) and fixed effects (FE). The insights for the panel models with two-way fixed effects(TWFE) are based on \textcite{wooldridge2021two}, and for the panel models with interactive fixed effects (IFE) they are based on \textcite{bai2009panel} and \textcite{moon2015linear}. Each subsection is structured to present the model with the set of assumptions it has to meet, the estimator, its implementation, its asymptotic properties, and an estimator of its variance for finite sample inference. Each assumption is explained either before or after it is stated.\footnote{However, inference and the asymptotic properties of TWFE and IFE are omitted due to the time and page limit on this paper and for conciseness. See \textcite{wooldridge2021two}, \textcite{bai2009panel} and references therein.}  \\
For the most part, I adapt the notation of \textcite{bonhomme2015grouped} and align the notation of other models accordingly. The main focus is on the novel method GFE which is why its theoretical properties is discussed in more detail in the next section. %$\theta$ denotes the true parameter vector for observed variables in all three models, but the notation for error term and for unobserved variables change as their structure change. 
All vectors are column vectors, transposed vectors are annotated by $'$, the Euclidean norm of a vector is denoted by $\lVert \cdot \rVert $, and all variables as well as parameters are real-valued.

%All vectors are column vectors and denoted by lowercase letters. Matrices are denoted by uppercase letters and $A'$ denotes the transpose of the matrix $A$. $\lVert . \rVert $ denotes the Euclidean norm of a vector.
